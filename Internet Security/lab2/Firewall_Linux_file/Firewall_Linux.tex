
%\newpage
%\setcounter{page}{1}
%\setcounter{section}{0}

\documentclass[11pt]{article}

\usepackage{times}
\usepackage{epsf}
\usepackage{epsfig}
\usepackage{amsmath, alltt, amssymb, xspace}
\usepackage{wrapfig}
\usepackage{fancyhdr}
\usepackage{url}
\usepackage{verbatim}
\usepackage{fancyvrb}

\usepackage{subfigure}
\usepackage{cite}
%\usepackage{cases}
%\usepackage{ltexpprt}
%\usepackage{verbatim}

%\topmargin      -0.70in  % distance to headers
%\headheight     0.2in   % height of header box
%\headsep        0.4in   % distance to top line
%\footskip       0.3in   % distance from bottom line

% Horizontal alignment
\topmargin      -0.50in  % distance to headers
\oddsidemargin  0.0in
\evensidemargin 0.0in
\textwidth      6.5in
\textheight     8.9in 


%\centerfigcaptionstrue

%\def\baselinestretch{0.95}


\newcommand\discuss[1]{\{\textbf{Discuss:} \textit{#1}\}}
%\newcommand\todo[1]{\vspace{0.1in}\{\textbf{Todo:} \textit{#1}\}\vspace{0.1in}}
\newtheorem{problem}{Problem}[section]
%\newtheorem{theorem}{Theorem}
%\newtheorem{fact}{Fact}
\newtheorem{define}{Definition}[section]
%\newtheorem{analysis}{Analysis}
\newcommand\vspacenoindent{\vspace{0.1in} \noindent}

%\newenvironment{proof}{\noindent {\bf Proof}.}{\hspace*{\fill}~\mbox{\rule[0pt]{1.3ex}{1.3ex}}}
%\newcommand\todo[1]{\vspace{0.1in}\{\textbf{Todo:} \textit{#1}\}\vspace{0.1in}}

%\newcommand\reducespace{\vspace{-0.1in}}
% reduce the space between lines
%\def\baselinestretch{0.95}

\newcommand{\fixmefn}[1]{ \footnote{\sf\ \ \fbox{FIXME} #1} }
\newcommand{\todo}[1]{
\vspace{0.1in}
\fbox{\parbox{6in}{TODO: #1}}
\vspace{0.1in}
}

\newcommand{\mybox}[1]{
\vspace{0.2in}
\noindent
\fbox{\parbox{6.5in}{#1}}
\vspace{0.1in}
}


\newcounter{question}
\setcounter{question}{1}

\newcommand{\myquestion} {{\vspace{0.1in} \noindent \bf Question \arabic{question}:} \addtocounter{question}{1} \,}

\newcommand{\myproblem} {{\noindent \bf Problem \arabic{question}:} \addtocounter{question}{1} \,}


\newcommand{\copyrightnoticeA}[1]{
\vspace{0.1in}
\fbox{\parbox{6in}{\small Copyright \copyright\ 2006 - 2014\ \ Wenliang Du, Syracuse University.\\ 
      The development of this document is partially funded by 
      the National Science Foundation's Course, Curriculum, and Laboratory 
      Improvement (CCLI) program under Award No. 0618680 and 0231122. 
      Permission is granted to copy, distribute and/or modify this document
      under the terms of the GNU Free Documentation License, Version 1.2
      or any later version published by the Free Software Foundation.
      A copy of the license can be found at http://www.gnu.org/licenses/fdl.html.}}
\vspace{0.1in}
}


\newcommand{\copyrightnotice}[1]{
\vspace{0.1in}
\fbox{\parbox{6in}{\small Copyright \copyright\ 2006 - 2014\ \ Wenliang Du, Syracuse University.\\
      The development of this document is/was funded by three grants from
      the US National Science Foundation: Awards No. 0231122 and 0618680 from
      TUES/CCLI and  Award No. 1017771 from Trustworthy Computing.
      Permission is granted to copy, distribute and/or modify this document
      under the terms of the GNU Free Documentation License, Version 1.2
      or any later version published by the Free Software Foundation.
      A copy of the license can be found at http://www.gnu.org/licenses/fdl.html.}}
\vspace{0.1in}
}

\newcommand{\copyrightnoticeB}[1]{
\vspace{0.1in}
\fbox{\parbox{6in}{\small Copyright \copyright\ 2006 - 2014\ \ Wenliang Du, Syracuse University.\\
      The development of this document is/was funded by the following grants from
      the US National Science Foundation: No. 0231122, 0618680, and 1303306.
      Permission is granted to copy, distribute and/or modify this document
      under the terms of the GNU Free Documentation License, Version 1.2
      or any later version published by the Free Software Foundation.
      A copy of the license can be found at http://www.gnu.org/licenses/fdl.html.}}
\vspace{0.1in}
}


\newcommand{\nocopyrightnotice}[1]{
\vspace{0.1in}
\fbox{\parbox{6in}{\small  
      The development of this document is funded by 
      the National Science Foundation's Course, Curriculum, and Laboratory 
      Improvement (CCLI) program under Award No. 0618680 and 0231122. 
      Permission is granted to copy, distribute and/or modify this document.
      }}
\vspace{0.1in}
}

\newcommand{\idea}[1]{
\vspace{0.1in}
{\sf IDEA:\ \ \fbox{\parbox{5in}{#1}}}
\vspace{0.1in}
}

\newcommand{\questionblock}[1]{
\vspace{0.1in}
\fbox{\parbox{6in}{#1}}
\vspace{0.1in}
}


\newcommand{\minix}{{\tt Minix}\xspace}
\newcommand{\unix}{{\tt Unix}\xspace}
\newcommand{\linux}{{\tt Linux}\xspace}
\newcommand{\ubuntu}{{\tt Ubuntu}\xspace}
\newcommand{\selinux}{{\tt SELinux}\xspace}
\newcommand{\freebsd}{{\tt FreeBSD}\xspace}
\newcommand{\solaris}{{\tt Solaris}\xspace}
\newcommand{\windowsnt}{{\tt Windows NT}\xspace}
\newcommand{\setuid}{{\tt Set-UID}\xspace}
%\newcommand{\smx}{{\tt Smx}\xspace}
\newcommand{\smx}{{\tt Minix}\xspace}
\newcommand{\relay}{{\tt relay}\xspace}
\newcommand{\isys}{{\tt iSYS}\xspace}
\newcommand{\ilan}{{\tt iLAN}\xspace}
\newcommand{\iSYS}{{\tt iSYS}\xspace}
\newcommand{\iLAN}{{\tt iLAN}\xspace}
\newcommand{\iLANs}{{\tt iLAN}s\xspace}
\newcommand{\bochs}{{\tt Bochs}\xspace}

\newcommand\FF{{\mathcal{F}}}

\newcommand{\argmax}[1]{
\begin{minipage}[t]{1.25cm}\parskip-1ex\begin{center}
argmax
#1
\end{center}\end{minipage}
\;
}

\newcommand{\bm}{\boldmath}
\newcommand  {\bx}    {\mbox{\boldmath $x$}}
\newcommand  {\by}    {\mbox{\boldmath $y$}}
\newcommand  {\br}    {\mbox{\boldmath $r$}}


%\pagestyle{fancyplain}
%\lhead[\thepage]{\thesection}      % Note the different brackets!
%\rhead[\thesection]{SEED Laboratories}
%\lfoot[\fancyplain{}{}]{Syracuse University} 
%\cfoot[\fancyplain{}{}]{\thepage} 

\newcommand{\tstamp}{\today}   
%\lhead[\fancyplain{}{\thepage}]         {\fancyplain{}{\rightmark}}
%\chead[\fancyplain{}{}]                 {\fancyplain{}{}}
%\rhead[\fancyplain{}{\rightmark}]       {\fancyplain{}{\thepage}}
%\lfoot[\fancyplain{}{}]                 {\fancyplain{\tstamp}{\tstamp}}
%\cfoot[\fancyplain{\thepage}{}]         {\fancyplain{\thepage}{}}
%\rfoot[\fancyplain{\tstamp} {\tstamp}]  {\fancyplain{}{}}

\pagestyle{fancy}
%\lhead{\bfseries Computer Security Course Project}
\lhead{\bfseries SEED Labs}
\chead{}
\rhead{\small \thepage}
\lfoot{}
\cfoot{}
\rfoot{}

\usepackage{listings}
\usepackage{color}

\definecolor{dkgreen}{rgb}{0,0.6,0}
\definecolor{gray}{rgb}{0.5,0.5,0.5}
\definecolor{mauve}{rgb}{0.58,0,0.82}

\lstset{frame=tb,
  language=C,
  aboveskip=3mm,
  belowskip=3mm,
  showstringspaces=false,
  columns=flexible,
  basicstyle={\small\ttfamily},
  numbers=none,
  numberstyle=\tiny\color{gray},
  keywordstyle=\color{blue},
  commentstyle=\color{dkgreen},
  stringstyle=\color{mauve},
  breaklines=true,
  breakatwhitespace=true,
  tabsize=3
}




\lhead{\bfseries SEED Labs -- Linux Firewall Exploration Lab}

\begin{document}



\begin{center}
{\LARGE Linux Firewall Exploration Lab}
\end{center}

\copyrightnotice

\newcounter{task}
\setcounter{task}{1}
\newcommand{\tasks} {\bf {\noindent (\arabic{task})} \addtocounter{task}{1} \,}


\section{Overview}

The learning objective of this lab is for students to gain the 
insights on how firewalls work 
by playing with firewall software and implement a simplified 
packet filtering firewall.
Firewalls have several types; in this lab, we focus on two types,
the {\em packet filter} and application firewall. 
Packet filters act by inspecting the packets;
if a packet matches the packet filter's set of rules, the packet filter will 
either drop the packet or forward it, depending on what the rules say. 
Packet filters are usually {\em stateless}; they filter each packet based 
only on the information contained in that packet, without paying 
attention to whether a packet is part of an existing stream of traffic.
Packet filters often use a combination of the packet's source and 
destination address, its protocol, and, for TCP and UDP traffic, 
port numbers. Application firewall works at 
the application layer. A widely used application firewall 
is web proxy, which is primarily used for egress filtering of 
web traffic.  In this lab, students will play with both 
types of firewalls, and also through the implementation of some 
of the key functionalities, they can understand how 
firewalls work.



\paragraph{Note for Instructors.}
If the instructor plans to hold lab sessions for this lab, it is suggested 
that the following topics should be covered:
\begin{itemize}
\item Loadable kernel module.
\item The Netfilter mechanism.
\end{itemize}




\section{Lab Tasks}


\subsection{Task 1: Using Firewall}

\linux has a tool called {\tt iptables}, which is essentially a firewall.
It has a nice front end program called {\tt ufw}. In this task, 
the objective is to use {\tt ufw} to set up some firewall policies, and 
observe the behaviors of your system after the policies become effective.
You need to set up at least two VMs, one called Machine A, and other called 
Machine B. You run the firewall on your Machine A. Basically, we use 
{\tt ufw} as a personal firewall. Optionally, if you have more VMs, you can 
set up the firewall at your router, so it can protect a network, instead of 
just one single computer. After you set up the two VMs, you should perform
the following tasks: 
\begin{itemize}
\item Prevent A from doing telnet to Machine B.
\item Prevent B from doing telnet to Machine A.
\item Prevent A from visiting an external web site. You can choose any web
site that you like to block, but keep in mind, some web servers have multiple
IP addresses. 
\end{itemize}


You can find the manual of {\tt ufw} by typing {\tt "man ufw"} or search it
online. It is pretty straightforward to use. Please remember that the
firewall is not enabled by default, so you should run a command to 
specifically enable it. We also list some commonly used commands 
in Appendix~\ref{sec:cheatsheet}. 


Before start the task, go to default policy file {\tt /etc/default/ufw}. If 
{\tt DEFAULT\_INPUT\_POLICY} is {\tt DROP}, please change it to {\tt ACCEPT}.  
Otherwise, all the incoming traffic will be dropped by default.

\subsection{Task 2: How Firewall Works} 

The firewall you used in the previous task is a packet filtering 
type of firewall. The main part of this type of firewall is the filtering part, 
which inspects each incoming and outgoing packets, and enforces the firewall policies 
set by the administrator. Since the packet 
processing is done within the kernel, the filtering must also be 
done within the kernel. Therefore, it seems that implementing such
a firewall requires us to modify the \linux kernel. In the past, 
this has to be done by modifying the kernel
code, and rebuild the entire kernel image. The modern \linux 
operating system provides several new mechanisms 
to facilitate the manipulation of packets without requiring the 
kernel image to be rebuilt. These two mechanisms are 
{\em Loadable Kernel Module} ({\tt LKM}) and {\tt Netfilter}.
 

{\tt LKM} allows us to add a new module to the kernel on the runtime. 
This new module enables us to extend the functionalities of the kernel,
without rebuilding the kernel or even rebooting the computer. 
The packet filtering part of firewalls can be implemented 
as an LKM. However, this is not enough. In order for the filtering module to 
block incoming/outgoing packets, the module 
must be inserted into the packet processing path. 
This cannot be easily done in the past before 
the {\tt Netfilter} was introduced into the \linux.


{\tt Netfilter} is designed to facilitate the manipulation of 
packets by authorized users. {\tt Netfilter} achieves this 
goal by implementing a number of {\em hooks} in the 
\linux kernel. These hooks are inserted into various places, 
including the packet incoming and outgoing paths. 
If we want to manipulate the incoming packets, we simply
need to connect our own programs (within LKM) to the 
corresponding hooks. Once an incoming packet arrives, 
our program will be invoked. Our program can decide 
whether this packet should be blocked or not; moreover,
we can also modify the packets in the program.


In this task, you need to use LKM and {\tt Netfilter} to implement
the packet filtering module.  This module will fetch 
the firewall policies from a data structure, and use the 
policies to decide whether packets should be blocked or not.
To make your life easier, so you can focus on the filtering part, 
the core of firewalls, we allow you to hardcode your firewall policies 
in the program. You should support at least five different 
rules, including the ones specified in the previous task. 


\questionblock{
\myquestion What types of hooks does {\tt Netfilter} support, and what can
you do with these hooks? Please draw a diagram to show how packets 
flow through these hooks.

\myquestion Where should you place a hook for ingress filtering, and 
where should you place a hook for egress filtering?

\myquestion Can you modify packets using {\tt Netfilter}?
}

\paragraph{Optional:} 
A real firewall should support a dynamic configuration, i.e.,
the administrator can dynamically change the firewall policies. 
The firewall configuration tool runs in the user space, 
but it has to send the data to the kernel space, where your 
packet filtering module (a LKM) can get the 
data. The policies must be stored in the kernel memory. 
You cannot ask your LKM to get the policies 
from a file, because that will significantly slow down 
your firewall. 
This involves interactions between a user-level program 
and the kernel module, which is not very difficult to implement. 
We have provided some guidelines in Section~\ref{sec:guidelines}; we also 
linked some tutorials in the web site. 
Implementing the dynamic configuration is optional. It is up to your 
instructor to decide whether you will receive bonus points for this
optional task.






\subsection{Task 3: Evading Egress Filtering}

Many companies and schools enforce egress filtering, which blocks users
inside of their networks from reaching out to certain web sites or Internet
services. They do allow users to access other web sites. 
In many cases, this type of firewalls inspect 
the destination IP address and port number in the outgoing packets. If 
a packet matches the restrictions, it will be dropped. 
They usually do not conduct deep packet inspections (i.e., looking into
the data part of packets) due to the performance reason. 
In this task, we show how such egress filtering can be bypassed using
the tunnel mechanism. There are many ways to establish tunnels; 
in this task, we only focus on SSH tunnels.

You need two VMs A and B for this task (three will be better). Machine A is running 
behind a firewall (i.e., inside the company or school's network), and Machine B 
is outside of the firewall.
Typically, there is a dedicated machine that runs the firewall, but
in this task, for the sake of convenience, you can run the firewall 
on Machine A.
You can use the firewall program that you implemented in the previous task,
or directly use {\tt ufw}.
You need to set up the following two block rules:
\begin{itemize}

\item Block all the outgoing traffic to external telnet servers.  In
reality, the servers that are blocked are usually game servers or other
types of servers that may affect the productivity of employees. In this 
task, we use the telnet server for demonstration purposes. You can run
the telnet server on Machine B (run {\tt sudo service openbsd-inetd
start}). If you have a third VM, Machine C, you can run the 
telnet server on Machine C. 
\begin{comment}
You can use the following command to set up the firewall rule; 
after that, try to telnet to Machine B (or C), and report your observation.
\begin{Verbatim}[frame=single] 
$ sudo ufw deny out port 23
\end{Verbatim}
\end{comment}


\item Block all the outgoing traffic to {\tt www.facebook.com}, so employees 
(or school kids) are not distracted during their work/school hours. Social
network sites are commonly blocked by companies and schools. After 
you set up the firewall, launch your Firefox browser, and try to connect to Facebook, and report
what happens. If you have already visited Facebook before using this 
browser, you need to clear all the caches using 
Firefox's menu: {\tt History -> Clear Recent History}; otherwise, the cached
pages may be displayed. If everything is set up properly, you 
should not be able to see the Facebook pages. 
It should be noted that Facebook has many IP addresses, it can 
change over the time. Remember to check whether the address is 
still the same by using {\tt ping} or {\tt dig} command. If 
the address has changed, you need to update your firewall rules. 
You can also choose web sites with static IP addresses, instead of using Facebook.
For example, most universities' web servers use static IP addresses (e.g. 
{\tt www.syr.edu}); for demonstration purposes, 
you can try block these IPs, instead of Facebook. 

\begin{comment}
You can use the following command to set up the firewall rule:
\begin{Verbatim}[frame=single] 
$ sudo ufw deny out to Facebook's IP
\end{Verbatim}
\end{comment}


\end{itemize}


In addition to set up the firewall rules, you also need the following
commands to manage the firewall: 
\begin{Verbatim}[frame=single] 
$ sudo ufw enable          // this will enable the firewall. 
$ sudo ufw disable         // this will disable the firewall. 
$ sudo ufw status numbered // this will display the firewall rules. 
$ sudo ufw delete 3        // this will delete the 3rd rule.
\end{Verbatim}




\paragraph{Task 3.a: Telnet to Machine B through the firewall}




To bypass the firewall, we can establish an SSH tunnel between
Machine A and B, so all the telnet traffic will go through this tunnel
(encrypted), evading the inspection. The following command 
establish an SSH tunnel from the localhost's port 8000 and 
machine B, and when packets come out of B's end, it will
be forwarded to Machine C's port 23 (telnet port). If you only have two VMs,
you can replace Machine C with Machine B.

\begin{Verbatim}[frame=single] 
$ ssh -L 8000:Machine_C_IP:23  seed@Machine_B_IP
\end{Verbatim}


\begin{figure}[htb]
        \centering
        \includegraphics*[viewport=124 447 580 691, width=0.60\textwidth]{Figs/SSHTunnel.pdf}
        \caption{SSH Tunnel Example}
        \label{fig:sshtunnel}
\end{figure}


After establishing the above tunnel, you can telnet to your localhost
using port 8000: {\tt telnet localhost 8000}.
SSH will transfer all your TCP packets from
your end of the tunnel (localhost:8000) to Machine B, and from there,
the packets will be forwarded to Machine\_C:23. Replies from Machine C will
take a reverse path, and eventually reach your telnet client.
This resulting in your telneting to Machine C.
Please describe your observation and explain how you are able to 
bypass the egress filtering. You should use Wireshark to see
what exactly is happening on the wire.
%\begin{Verbatim}[frame=single] 
%$ telnet localhost 8000
%\end{Verbatim}





\paragraph{Task 3.b: Connecting to Facebook using SSH Tunnel.}
To achieve this goal, we can use the approach similar to that in 
Task 3.a, i.e., establishing a tunnel between your localhost:port
and Machine B, and ask B to forward packets to Facebook. To do 
that, you can use the following command to set up the tunnel:
{\tt "ssh -L 8000:FacebookIP:80 ..."}. 
We will not use this approach, and instead, we 
use a more generic approach, called dynamic port forwarding, instead of a static one
like that in Task 3.a. To do that, we only specify the local
port number, not the final destination. When Machine B receives
a packet from the tunnel, it will dynamically decide where it should 
forward the packet to based on the destination information of
the packet.
\begin{Verbatim}[frame=single] 
$ ssh -D 9000 -C seed@machine_B
\end{Verbatim}


Similar to the telnet program, which connects {\tt localhost:9000}, 
we need to ask Firefox to connect to {\tt localhost:9000} every time it 
needs to connect to a web server, so the traffic can 
go through our SSH tunnel. To achieve that, we can tell Firefox to
use {\tt localhost:9000} as its proxy. The following procedure
achieves this:
\begin{Verbatim}[frame=single] 
Edit -> Preference -> Advanced tab -> Network tab -> Settings button.

Select Manual proxy configuration
SOCKS Host: 127.0.0.1      Port: 9000
SOCKS v5
No Proxy for: localhost, 127.0.0.1
\end{Verbatim}

After the setup is done, please do the followings:
\begin{enumerate}
\item Run Firefox and go visit the Facebook page.
Can you see the Facebook page? 
Please describe your observation. 

\item After you get the facebook page, break the SSH tunnel, 
clear the Firefox cache, and try the connection again. 
Please describe your observation. 

\item Establish the SSH tunnel again and connect to Facebook. 
Describe your observation. 

\item Please explain what you have observed, especially
on why the SSH tunnel can help bypass the egress filtering. 
You should use Wireshark to see
what exactly is happening on the wire. Please describe your 
observations and explain them using the packets that you
have captured.
\end{enumerate}

\questionblock{\myquestion If {\tt ufw} blocks the TCP port 22, which
is the port used by SSH, can you still set up an SSH tunnel to evade 
egress filtering?}



\subsection{Task 4: Web Proxy (Application Firewall)}


There is another type of firewalls, which are specific to 
applications. Instead of inspecting the packets at 
the transport layer (such as TCP/UDP) and below (such as IP), 
they look at the application-layer data, and enforce 
their firewall policies. These firewalls are called 
application firewalls,  which controls input, output,
and/or access from, to, or by an application or service.
A widely used category of application firewalls is web proxy. 
which is used to control what their protected browsers can
access. This is a typical egress filtering, and it is widely
used by companies and schools to block their employees 
or students from accessing distracting or inappropriate 
web sites. 


In this task, we will set up a web proxy and perform some 
tasks based on this web proxy. There are a number of 
web proxy products to choose from. In this lab, we will 
use a very well-known free software, called {\tt squid}. If you 
are using our pre-built VM (Ubuntu 12.04), this software is already installed. 
Otherwise, you can easily run the following command to 
install it.
\begin{Verbatim}[frame=single] 
$ sudo apt-get install squid

Here are several commands that you may need:

$ sudo service squid3 start     // to start the server
$ sudo service squid3 restart   // to restart the server
\end{Verbatim}


Once you have installed {\tt squid}, you can go to {\tt /etc/squid3},
and locate the configuration file called {\tt squid.conf}. This is 
where you need to set up your firewall policies. Keep in mind that 
every time you make a change to {\tt squid.conf}, you need to 
restart the {\tt squid} server; otherwise, your changes will 
not take effect.


\paragraph{Task 4.a: Setup.} You need to set up two VMs, Machine A 
and Machine B. Machine A is the one whose browsing behaviors need to restricted,
and Machine B is where you run the web proxy. We need to configure
A's Firefox browser, so it always use the web proxy server on B.
To achieve that, we can tell Firefox to
use {\tt B:3128} as its proxy (by default, {\tt squid} uses 
port {\tt 3128}, but this can be changed in {\tt squid.conf}).
The following procedure
configures {\tt B:3128} as the proxy for Firefox.
\begin{Verbatim}[frame=single] 
Edit -> Preferences -> Advanced tab -> Network tab -> Settings button.

Select "Manual proxy configuration"
Fill in the following information:
HTTP Proxy: B's IP address      Port: 3128
\end{Verbatim}

Note: to ensure that the browser always uses the proxy server, 
the browser's proxy setting needs to be locked down, so users cannot 
change it. There are ways for administrators to do that. If you are 
interested, you can search the Internet for instructions.


After the setup is done, please perform the following tasks:
\begin{enumerate}
\item Try to visit some web sites from Machine A's
Firefox browser, and describe your observation. 

\item By default, all the external web sites are blocked. 
Please Take a look at the configuration file
{\tt squid.conf}, and describe what rules have caused that (hint: 
search for the {\tt http\_access} tag in the file).

\item Make changes to the configuration file, so 
all the web sites are allowed. 

\item Make changes to the configuration file, so only
the access to {\tt google.com} is allowed. 
\end{enumerate}

You should turn on your Wireshark, capture packets while 
performing the above tasks. In your report, you should describe 
what exactly is happening on the wire. Using these observations, you
should describe how the web proxy works.


\paragraph{Task 4.b: Using Web Proxy to evade Firewall.}
Ironically, web proxy, which is widely used to do the egress 
filtering, is also widely used to bypass egress filtering.
Some networks have packet-filter type of firewall, which 
blocks outgoing packets by looking at their destination
addresses and port numbers. For example, in Task 1,
we use {\tt ufw} to block the access of Facebook. 
In Task 3, we have shown that you can use a SSH tunnel to 
bypass that kind of firewall. In this task, 
you should do it using a web proxy. Please 
demonstrate how you can access Facebook even if {\tt ufw}
has already blocked it. Please describe how you do that and 
also include evidence of success in your lab report.


\questionblock{\myquestion If {\tt ufw} blocks the TCP port 3128, can 
you still use web proxy to evade the firewall?}




\paragraph{Task 4.c: URL Rewriting/Redirection.}

Not only can {\tt squid} block web sites, they can also 
rewrite URLs or redirect users to another web site. 
For example, you can set up {\tt squid}, so every time a user
tries to visit Facebook, you redirect them to a web page, 
which shows a big red stop sign. {\tt Squid} allows you 
to make any arbitrary URL rewriting: all you need to do is to 
connect {\tt squid} to an URL rewriting program. In 
this task, you will write a very simple URL-rewriting 
programming to demonstrate such a functionality of 
web proxy. You can use any programming language, such as 
Perl, Java, C/C++, etc.

Assume that you have written a program called {\tt myprog.pl} (a perl 
program). You need to modify {\tt squid.conf} to connect
to this program. You need to add the following 
to the configuration file: 
\begin{Verbatim}[frame=single] 
url_rewrite_program /home/seed/myprog.pl
url_rewrite_children 5
\end{Verbatim}

In case {\tt squid} cannot find {\tt myprog.pl} under {\tt /home/seed},
you can put your url rewrite program under {\tt /etc/squid3}. 

This is how it works: when {\tt squid} receives an URL from browsers,
it will invoke {\tt myprog.pl}, which can get the URL information
from the standard input (the pipe mechanism is used). 
The rewriting program can
do whatever it wants to the URL, and eventually print out a new 
URL or an empty line to the standard output, which is then piped back to {\tt squid}. 
{\tt Squid} uses this output as the new URL, if the line is not empty. 
This is how an URL gets rewritten. 
The following Perl program gives you an example:
\begin{Verbatim}[frame=single] 
#!/usr/bin/perl -w
use strict;
use warnings;

# Forces a flush after every write or print on the STDOUT
select STDOUT; $| = 1;

# Get the input line by line from the standard input.
# Each line contains an URL and some other information.
while (<>)
{
    my @parts = split;  
    my $url = $parts[0];
    # If you copy and paste this code from this PDF file, 
    # the ~ (tilde) character may not be copied correctly. 
    # Remove it, and then type the character manually.
    if ($url =~ /www\.cis\.syr\.edu/) {  
        # URL Rewriting
        print "http://www.yahoo.com\n";
    }
    else { 
        # No Rewriting. 
        print "\n";
    }
}
\end{Verbatim}



Please conduct the following tasks:
\begin{enumerate}
\item Please describe what the above program does (e.g. what URLs got rewritten
and what your observations are).

\item Please modify the above program, so it replaces all the Facebook pages with a 
page that shows a big red stop sign. 

\item Please modify the above program, so it replaces all the images (e.g.
{\tt .jpg} or {\tt .gif} images) inside any page with a picture of your choice. 
When an HTML web page contains images, the browser will identify those
image URLs, and send out an URL request for each image. You should be able
to see the URLs in your URL
rewriting program. You just need to decide whether a URL is trying to
fetch an image file of a particular type; if it is, you can replace the URL
with another one (of your choice). 


\end{enumerate}


\paragraph{Note:} If you have a syntax error in the above program, you will not be able to see 
the error message if you test it via the web proxy. We suggest that you 
run the above perl program on the command line first, manually
type an URL as the input, and see whether the program functions
as expected. If there is a syntax error, you will see the error message.
Do watch out for the tilde sign if you copy and paste the above
program from the PDF file.



\paragraph{Task 4.d (Optional): A Real URL Redirector.} 
The program you wrote above is a toy URL rewriting program. In reality,
such a program is much more complicated because of performance issues
and many other functionalities. Moreover, they work with 
real URL datasets (the URLs that need to be blocked and rewritten) that
contains many URLs. These datasets can be obtained 
from various sources (some are free and some are not), and they
need to be routinely updated. If you are interested in playing with 
a real URL rewriting program, you can install {\tt SquidGuard} and 
also download a blacklist URL dataset that works with it. Here is some 
related information:
\begin{itemize}
\item You can download {\tt SquidGuard} from
\url{http://www.squidguard.org/}, compile and install it.
You can get the instruction from the following URL
\url{http://www.squidguard.org/Doc/}.

\item You also need to download Oracle Berkeley DB, because 
{\tt SquidGuard} depends on it. You should do this step first.

\item Download a blacklist data set from the following URL:

\url{http://www.squidguard.org/blacklists.html}.

\end{itemize}









\section{Guidelines}
\label{sec:guidelines}

\subsection{Loadable Kernel Module}

The following is a simple loadable kernel module. It prints out 
{\tt "Hello World!"} when the module is loaded; when the module
is removed from the kernel, it prints out {\tt "Bye-bye World!"}.
The messages are not printed out on the screen; they are 
actually printed into the {\tt /var/log/syslog} file. You can
use {\tt dmesg | tail -10} to read the last 10 lines of message.

\begin{Verbatim}[frame=single] 
#include <linux/module.h>
#include <linux/kernel.h>

int init_module(void)
{
        printk(KERN_INFO "Hello World!\n");
        return 0;
}

void cleanup_module(void)
{
        printk(KERN_INFO "Bye-bye World!.\n");
}
\end{Verbatim}

\noindent
We now need to create {\tt Makefile}, which includes the following
contents (the above program is named {\tt hello.c}). Then 
just type {\tt make}, and the above program will be compiled
into a loadable kernel module.


\begin{Verbatim}[frame=single] 
obj-m += hello.o

all:
        make -C /lib/modules/$(shell uname -r)/build M=$(PWD) modules

clean:
        make -C /lib/modules/$(shell uname -r)/build M=$(PWD) clean
\end{Verbatim} 



\noindent
Once the module is built by typing {\tt make}, you can use the following commands to 
load the module, list all modules, and remove the module:

\begin{Verbatim}[frame=single] 
 % sudo insmod mymod.ko        (inserting a module)
 % lsmod                       (list all modules)
 % sudo rmmod mymod.ko         (remove the module)
\end{Verbatim} 

Also, you can use {\tt modinfo mymod.ko} to show information about a 
Linux Kernel module.


\subsection{Interacting with Loadable Kernel Module (for the Optional Task)}

In firewall implementation, the packet filtering part is implemented
in the kernel, but the policy setting is done at the user space.
We need a mechanism to pass the policy information from a user-space 
program to the kernel module. There are several ways to 
do this; a standard approach is to use {\tt /proc}.
Please read the article from 
\url{http://www.ibm.com/developerworks/linux/library/l-proc.html} 
for detailed instructions. Once we set up a {\tt /proc} file 
for our kernel module, we can use the standard {\tt write()} and {\tt read()}
system calls to pass data to and from the kernel module. 

\begin{Verbatim}[frame=single]
#include <linux/module.h>
#include <linux/kernel.h>
#include <linux/proc_fs.h>
#include <linux/string.h>
#include <linux/vmalloc.h>
#include <asm/uaccess.h>

MODULE_LICENSE("GPL");
MODULE_DESCRIPTION("Fortune Cookie Kernel Module");
MODULE_AUTHOR("M. Tim Jones");

#define MAX_COOKIE_LENGTH       PAGE_SIZE

static struct proc_dir_entry *proc_entry;
static char *cookie_pot;  // Space for fortune strings
static int cookie_index;  // Index to write next fortune
static int next_fortune;  // Index to read next fortune

ssize_t fortune_write( struct file *filp, const char __user *buff,
                        unsigned long len, void *data );

int fortune_read( char *page, char **start, off_t off,
                   int count, int *eof, void *data );

int init_fortune_module( void )
{
  int ret = 0;

  cookie_pot = (char *)vmalloc( MAX_COOKIE_LENGTH );

  if (!cookie_pot) {
    ret = -ENOMEM;
  } else {
    memset( cookie_pot, 0, MAX_COOKIE_LENGTH );
    proc_entry = create_proc_entry( "fortune", 0644, NULL );
    if (proc_entry == NULL) {
      ret = -ENOMEM;
      vfree(cookie_pot);
      printk(KERN_INFO "fortune: Couldn't create proc entry\n");
    } else {
      cookie_index = 0;
      next_fortune = 0;
      proc_entry->read_proc = fortune_read;
      proc_entry->write_proc = fortune_write;

      printk(KERN_INFO "fortune: Module loaded.\n");
    }
  }

  return ret;
}

void cleanup_fortune_module( void )
{
  remove_proc_entry("fortune", NULL);
  vfree(cookie_pot);
  printk(KERN_INFO "fortune: Module unloaded.\n");
}

module_init( init_fortune_module );
module_exit( cleanup_fortune_module );
\end{Verbatim}

The function to read a fortune is shown as following:

\begin{Verbatim}[frame=single]
int fortune_read( char *page, char **start, off_t off,
                   int count, int *eof, void *data )
{
  int len;

  if (off > 0) {
    *eof = 1;
    return 0;
  }

  /* Wrap-around */
  if (next_fortune >= cookie_index) next_fortune = 0;
  len = sprintf(page, "%s\n", &cookie_pot[next_fortune]);
  next_fortune += len;

  return len;
}
\end{Verbatim}

The function to write a fortune is shown as following. Note that
we use $copy\_from\_user$ to copy the user buffer directly into the $cookie\_pot$.

\begin{Verbatim}[frame=single]
ssize_t fortune_write( struct file *filp, const char __user *buff,
                        unsigned long len, void *data )
{
  int space_available = (MAX_COOKIE_LENGTH-cookie_index)+1;

  if (len > space_available) {
    printk(KERN_INFO "fortune: cookie pot is full!\n");
    return -ENOSPC;
  }

  if (copy_from_user( &cookie_pot[cookie_index], buff, len )) {
    return -EFAULT;
  }

  cookie_index += len;
  cookie_pot[cookie_index-1] = 0;
  return len;
}
\end{Verbatim}



\subsection{A Simple Program that Uses Netfilter}

Using {\tt Netfilter} is quite straightforward. All we need to do
is to hook our functions (in the kernel module) to the corresponding
{\tt Netfilter} hooks. Here we show an example:

\begin{Verbatim}[frame=single] 
#include <linux/module.h>
#include <linux/kernel.h>
#include <linux/netfilter.h>
#include <linux/netfilter_ipv4.h>

/* This is the structure we shall use to register our function */
static struct nf_hook_ops nfho;

/* This is the hook function itself */
unsigned int hook_func(unsigned int hooknum,
                       struct sk_buff *skb,
                       const struct net_device *in,
                       const struct net_device *out,
                       int (*okfn)(struct sk_buff *))
{
    /* This is where you can inspect the packet contained in
       the structure pointed by skb, and decide whether to accept 
       or drop it. You can even modify the packet */

    // In this example, we simply drop all packets
    return NF_DROP;           /* Drop ALL packets */
}

/* Initialization routine */
int init_module()
{   /* Fill in our hook structure */
    nfho.hook = hook_func;         /* Handler function */
    nfho.hooknum  = NF_INET_PRE_ROUTING; /* First hook for IPv4 */
    nfho.pf       = PF_INET;
    nfho.priority = NF_IP_PRI_FIRST;   /* Make our function first */

    nf_register_hook(&nfho);
    return 0;
}

/* Cleanup routine */
void cleanup_module()
{
    nf_unregister_hook(&nfho);
}
\end{Verbatim} 

When compiling some of the examples from the tutorial, you might see an error that says
that\\ {\tt NF\_IP\_PRE\_ROUTING} is undefined. Most likely, this example
is written for the older \linux kernel. Since version 2.6.25,
kernels have been using {\tt NF\_INET\_PRE\_ROUTING}. Therefore, 
replace {\tt NF\_IP\_PRE\_ROUTING} with {\tt NF\_INET\_PRE\_ROUTING}, 
this error will go away (the replacement is already done in the code above).


\subsection{The Lightweight Firewall Code}

Owen Klan wrote a very nice lightweight firewall program, which we have linked
in this lab's web page ({\tt lwfw.tar.gz}). From this sample code, you can see
how to write a simple hook for Netfilter, and how to retrieve the IP and 
TCP/UDP headers inside the hook. You can start with this program,
make it work, and gradually expand it to support more sophisticated firewall
policies (this sample program only enforces a very simple policy).


\begin{comment}
\subsection{Parsing Command Line Arguments}

Because our \minifirewall program needs to recognize
command line arguments, we need to 
parse these arguments. If the syntax for the command line arguments is 
simple enough, we can directly write code to parse them. However,
our \minifirewall has to recognized options with a 
a fairly sophisticated syntax. We can use 
{\tt getopt()} and {\tt getopt\_long()} to 
systematically parse command line arguments.
Please read the tutorial in the following URL:
\begin{quote}
\url{http://www.gnu.org/software/libc/manual/html_node/Getopt.html}
\end{quote}
\end{comment}


\section{Submission and Demonstration}


Students need to submit a detailed lab report to describe what they have
done, what they have observed, and explanation. Reports should include the evidences to support
the observations. Evidences include packet traces, screendumps, etc.
Students also need to answer all the questions in the lab description.
For the programming tasks, students should list the important code snippets followed by
explanation. Simply attaching code without any explanation is not enough.



% Note: in the manual, don't use the question index, instead, repeat the
% question, because the question index might change when we revise the lab
% description and forget to change the manual. 

\vspace{.2in}

%\myquestion: Please convert the 32-bit IP address {\tt "128.230.10.1"} to 
%an integer of the network byte order, and also as an integer of the 
%host byte order (please tell us what type of CPU you have on your machine). 

\questionblock{
\myquestion We can use the SSH and HTTP protocols as tunnels to evade the egress
filtering. Can we use the ICMP protocol as a tunnel to evade the egress filtering? 
Please briefly describe how.
}



\newpage
\appendix

\section{Firewall Lab Cheat Sheet} 
\label{sec:cheatsheet}

\paragraph{Header Files.} You may need to take a look at several header
files, including the {\tt skbuff.h}, {\tt ip.h}, {\tt icmp.h}, 
{\tt tcp.h}, {\tt udp.h}, and {\tt netfilter.h}. They are stored in
the following folder: 
\begin{Verbatim}[frame=single]
/lib/modules/$(uname -r)/build/include/linux/
\end{Verbatim}


\paragraph{IP Header.}
The following code shows how you can get the IP header, and its 
source/destination IP addresses. 
\begin{Verbatim}[frame=single]
struct iphdr *ip_header = (struct iphdr *)skb_network_header(skb);
unsigned int src_ip = (unsigned int)ip_header->saddr;
unsigned int dest_ip = (unsigned int)ip_header->daddr;
\end{Verbatim}

\paragraph{TCP/UDP Header.}
The following code shows how you can get the UDP header, and its 
source/destination port numbers. It should be noted that we 
use the {\tt ntohs()} function to convert the unsigned short integer 
from the network byte order to the host byte order. This is because
in the 80x86 architecture, the host byte order is the Least Significant Byte first, 
whereas the network byte order, as used on the Internet, is Most Significant Byte
first. If  you want to put a short integer into a packet, 
you should use {\tt htons()}, which is reverse to {\tt ntohs()}. 
\begin{Verbatim}[frame=single]
struct udphdr *udp_header = (struct udphdr *)skb_transport_header(skb);  
src_port = (unsigned int)ntohs(udp_header->source);        
dest_port = (unsigned int)ntohs(udp_header->dest);    
\end{Verbatim}


\paragraph{IP Addresses in different formats.}
You may find the following library functions useful when you convert
IP addresses from one format to another (e.g. from a string {\tt "128.230.5.3"}
to its corresponding integer in the network byte order or the host byte order.
\begin{Verbatim}[frame=single]
int inet_aton(const char *cp, struct in_addr *inp);
in_addr_t inet_addr(const char *cp);
in_addr_t inet_network(const char *cp);
char *inet_ntoa(struct in_addr in);
struct in_addr inet_makeaddr(int net, int host);
in_addr_t inet_lnaof(struct in_addr in);
in_addr_t inet_netof(struct in_addr in);
\end{Verbatim}


\paragraph{Using {\tt ufw}.}
The default firewall configuration tool for Ubuntu is {\tt ufw}, 
which is developed to ease {\tt iptables} firewall configuration. 
By default UFW is disabled, so you need to enable it first.
\begin{Verbatim}[frame=single]
$ sudo ufw enable            // Enable the firewall
$ sudo ufw disable           // Disable the firewall
$ sudo ufw status numbered   // Display the firewall rules
$ sudo ufw delete 2          // Delete the 2nd rule
\end{Verbatim}

\paragraph{Using {\tt squid}.} The following commands are related to {\tt
squid}.
\begin{Verbatim}[frame=single]
$ sudo service squid3 start         // start the squid service
$ sudo service squid3 restart       // restart the squid service
$ sudo service squid3 stop          // stop the squid service

/etc/squid3/squid.conf:  This is the squid configuration file. 
\end{Verbatim}


\end{document}
